\documentclass[a4paper, 12pt]{book}

\usepackage[italian]{babel}

\linespread{1.3}



\title{Sintesi di Composti d'interesse Medico per il Trattamento del Morbo d'Alzheimer}
\author{
	Relatore: Annamaria DeAgostino
	\and
	Candidato: Lorenzo Castellino
}
\date{Anno Accademico 2018-2019}

\begin{document}
\pagenumbering{gobble}
\maketitle
\setcounter{page}{0}
\tableofcontents
\newpage
\pagenumbering{arabic}

\chapter*{Introduzione al Morbo d'Alzheimer}
\addcontentsline{toc}{section}{Introduzione al Morbo d'Alzheimer}
\markboth{INTRODUZIONE}{INTRODUZIONE}
La malattia di Alzheimer-Perusini, nota più comunemente come morbo d'Alzheimer (AD), è una tra le forme più diffuse al mondo di demenza senile. Dal punto di vista medico la patologia è definita come "disturbo neurocognitivo maggiore o lieve dovuto a malattia di Alzheimer".\cite{american_psychiatric_association_diagnostic_2013}
I sintomi associati sono differenti da individuo ad individuo, in generale nei pazienti riconosciuti si sono osservate:
\begin{enumerate}
	\item Perdita di memoria a breve termine.
	\item Difficoltà a concentrarsi, organizzarsi e di pianificazione.
	\item Difficoltà a seguire e/o formulare discorsi di senso compiuto.
	\item Difficoltà a giudicare distanze e spazi.
	\item Perdita di orientamento spaziale e temporale.
	\item Cambi repentini dell'umore.
	\item Allucinazioni visive.
\end{enumerate}
La demenza è una patologia di tipo progressivo, ovvero si ha un peggiormento dei sintomi col passare del tempo. La velocità di tale processo varia da persona a persona, tendenzialmente l'aspettativa media di vita successiva alla diagnosi della condizione va dai 3 ai 10 anni. \cite{todd_survival_2013}

\section{L'Importanza della Ricerca}
L'incidenza dell'AD è in aumento; stando al World Alzheimer Report del 2018, stilato dall'Alzheimer's Disease International ovvero l'associazione internazionale per la lotta all'Alzheimer in stretta collaborazione con la World Health Organization, si stima che nel mondo circa 50 milioni di persone siano affette da demenza. Un numero che per il 2050 potrebbe triplicarsi raggiungendo quindi i circa 150 milioni d'individui colpiti a livello globale.
I costi stimati per la gestione dei casi di demenza sono altrettanto preoccupanti con una spesa attuale di circa 1 miliardo di dollari l'anno che entro il 2050 potrebbe raddoppiare.\cite{noauthor_world_2018}





\bibliographystyle{plain}
\bibliography{biblio.bib}
\end{document}
