\documentclass[a4paper, 12pt]{book}

\usepackage[italian]{babel}
\usepackage{float}

\oddsidemargin 30pt
\evensidemargin 20pt
\linespread{1.3}


\title{Sintesi di Composti d'interesse Medico per il Trattamento del Morbo d'Alzheimer}
\author{
	Relatore: Annamaria DeAgostino
	\and
	Candidato: Lorenzo Castellino
}
\date{Anno Accademico 2018-2019}

\begin{document}
\pagenumbering{gobble}
\maketitle
\setcounter{page}{0}
\tableofcontents
\newpage
\pagenumbering{arabic}

\chapter*{Introduzione}
\addcontentsline{toc}{section}{Introduzione}
La malattia di Alzheimer-Perusini, nota più comunemente come morbo d'Alzheimer (AD), è una tra le forme più diffuse al mondo di demenza senile. Dal punto di vista medico la patologia è definita come "disturbo neurocognitivo maggiore o lieve dovuto a malattia di Alzheimer" (DMS-5).
I sintomi associati sono differenti da individuo ad individuo, in generale nei pazienti riconosciuti si sono osservate:
\begin{enumerate}
	\item Perdita di memoria a breve termine.
	\item Difficoltà a concentrarsi, organizzarsi e di pianificazione.
	\item Difficoltà a seguire e/o formulare discorsi di senso compiuto.
	\item Difficoltà a giudicare distanze e spazi.
	\item Perdita di orientamento spaziale e temporale.
	\item Cambi repentini dell'umore.
	\item Allucinazioni visive.
\end{enumerate}
La demenza è una patologia di tipo progressivo, ovvero si ha un peggiormento dei sintomi col passare del tempo



\end{document}
