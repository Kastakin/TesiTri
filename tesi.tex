\documentclass[a4paper, 12pt]{article}

\usepackage[italian]{babel}

\linespread{1.3}



\title{Sintesi di Composti d'interesse Medico per il Trattamento del Morbo d'Alzheimer}
\author{
	Relatore: Annamaria DeAgostino
	\and
	Candidato: Lorenzo Castellino
}
\date{Anno Accademico 2018-2019}

\begin{document}
\pagenumbering{gobble}
\maketitle
\setcounter{page}{0}
\newpage
\tableofcontents
\newpage
\pagenumbering{arabic}

\section{Introduzione al Morbo d'Alzheimer}
La malattia di Alzheimer-Perusini, nota più comunemente come morbo d'Alzheimer (AD), è una tra le forme più diffuse al mondo di demenza senile. Dal punto di vista medico la patologia è definita come "disturbo neurocognitivo maggiore o lieve dovuto a malattia di Alzheimer".\cite{american_psychiatric_association_diagnostic_2013}
I sintomi associati sono differenti da individuo ad individuo, in generale nei pazienti riconosciuti si sono osservate:
\begin{enumerate}
	\item Perdita di memoria a breve termine.
	\item Difficoltà a concentrarsi, organizzarsi e di pianificazione.
	\item Difficoltà a seguire e/o formulare discorsi di senso compiuto.
	\item Difficoltà a giudicare distanze e spazi.
	\item Perdita di orientamento spaziale e temporale.
	\item Cambi repentini dell'umore.
	\item Allucinazioni visive.
\end{enumerate}
La demenza è una patologia di tipo progressivo, ovvero si ha un peggiormento dei sintomi col passare del tempo. La velocità di tale processo varia da persona a persona, tendenzialmente l'aspettativa media di vita successiva alla diagnosi della condizione va dai 3 ai 10 anni. \cite{todd_survival_2013}

\subsection{L'Importanza della Ricerca}
L'incidenza dell'AD è in aumento tanto da individuare nella ricerca di una cura come una delle sfide per il nuovo millenio: stando al World Alzheimer Report del 2018, stilato dall'Alzheimer's Disease International ovvero l'associazione internazionale per la lotta all'Alzheimer in stretta collaborazione con la World Health Organization, si stima che nel mondo circa 50 milioni di persone siano affette da demenza. Ciò si traduce in una spesa annua per il trattamento dei malati che rasenta il miliardo di dollari.

Con l'aumento dell'aspettativa media di vita si prevede che nel 2050 il numero di casi sarà il triplo di quello odierno e si prospetta una spesa doppia rispetto a quella attuale.\cite{noauthor_world_2018}

Stando a queste previsioni 1 individuo su 85 nel 2050 sarà affetto da demenza.





\newpage

\bibliographystyle{plain}
\bibliography{biblio.bib}
\end{document}
